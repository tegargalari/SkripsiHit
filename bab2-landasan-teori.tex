%-----------------------------------------------------------------------------%
\chapter{\babDua}
%-----------------------------------------------------------------------------%
Pada bab ini akan dijelaskan teori - teori yang akan digunakan \saya dalam penelitian ini. Penjelasan teori yang terdapat dibab ini merupakan hasil dari pembelajaran \saya dari literatur maupun pengalaman yang telah \saya alami.
%-----------------------------------------------------------------------------%
\section{Teori Perancangan Game}
	\subsection{Definisi Game}
	\textit{Game} merupakan media interaksi yang memadukan beberapa elemen. Elemen yang dimaksud berupa gambar, tulisan, suara dan lain - lain. Menurut Rogers (2010) dalam bukunya yang berjudul "\textit{Level Up:The Guide To Great Video Game Design}", \textit{game} adalah aktivitas yang memiliki peraturan, tujuan, dan minimal satu pemain. Menurut Schell (2015) dalam buku "\textit{The Art of Game Design}", \textit{game} adalah "\textit{an exercise of voluntary control systems, in which there is a contest between powers, confined by rules in order to produce a disequilibrial outcome}".
	\linebreak\linebreak
	Menurut buku "\textit{Rules of Play}", Salen \& Zimmerman (2004), beberapa peneliti telah mengutarakan definisi dari \textit{game}. Salen \& Zimmerman mengatakan \textit{game} adalah sebuah konflik yang dibuat sedemikian rupa, terdapat peraturan didalamnya dan sebuah hasil. David Parlett mengatakan ada dua elemen penting yaitu \textit{Ends} (akhir dari  \textit{game} yang telah didefinisikan) dan \textit{Means} (cara seorang pemain untuk mencapai tujuan game tersebut).
	\subsection{Kategori Game}
	Jumlah game saat ini sudah meningkat setiap tahunnya. Setiap game memiliki ciri khas yang berbeda - beda. Untuk memudahkan dalam mengenali jenis \textit{game}, jurnalis, pemain, dan developer sepakat untuk mengklasifikasi \textit{game} sesuai dengan katagorinya. Herz (1997) mengelompokkan \textit{game} menjadi :
	\begin{itemize}
		\item Action Game
			\subitem Genre ini mengutamakan kemampuan fisik dari pemainnya. Kemampuan yang dituntut dalam memainkan genre ini berupa koordinasi mata dengan reflek dari pemainnya. Pemain akan menjadi pemeran utama yang akan melakukan begitu banyak aksi didalamnya.
		\item Role-Playing Game
			\subitem Sebuah genre dimana pemain akan memeran seorang karakter dalam \textit{game} yang memiliki sebuah cerita yang harus diselesaikan.
		\item Simulation Game
			\subitem Genre yang mengambil sebuah kejadian dari kehidupan nyata dan diubah menjadi bentuk \textit{game}. Sebagai contoh permainan mesimulasi sebagai batu, dalam \textit{game} tersebut pemain akan memerankan sebagai batu yang hanya bisa diam dan terombang - ambing.
		\item Strategy Game
			\subitem Sebuah Genre dimana pemain mengendalikan sebuah atau beberapa unit dan mengatur cara agar dapat memenangkan permainan tersebut.
		\item Sports Game
			\subitem Genre ini sejenis dengan simulasi, genre ini lebih memfokuskan tentang kejadian pada dunia olahraga.
		\item Idle Game
			\subitem Genre ini meminimkan aksi yang dilakukan oleh pemain. Contoh \textit{game} dari genre ini adalah "Cookie Clicker" yang hanya mengharuskan pemain untuk menekan layar pada perangkat kerasnya.
	\end{itemize}
	
	\subsection{Elemen Game}
	Terdapat beberapa elemen dalam \textit{game} yang sangat penting dan menjadi rujukan untuk meningkatkan performa dari permainan yang dibuat oleh developer. Schell (2015) menuliskan elemen yang ada dalam sebuah permainan dalam buku "\textit{The Art of Game Design} sebagai berikut :
	\begin{itemize}
		\item Estetika
			\subitem Elemen untuk menampilkan gambar, suara dan suasana dalam permainan tersebut. Dengan menampilkan hal - hal tersebut maka pengalaman user dalam memainkan permainan tersebut akan meningkat.
		\item Teknologi
			\subitem Elemen ini merupakan struktur bagaimana permainan ini dibuat. Dalam pengembangan \textit{game} 
		\item Mekanik
		\item Naratif
	\end{itemize}
	
\section{Teori Pembelajaran}
	\subsection{Definisi Pembelajaran}
	
\section{Pembelajaran Berbasis Game}
	\subsection{Definisi Pembelajaran Berbasis Game}
	Lorem Ipsum
	\subsection{Prinsip dan Penggunaan Game Sebagai Pembelajaran}
	\subsection{Karakteristik Desain Pembelajaran Berbasis Game}
	
\section{Fundamental Programming}
	\subsection{Definisi Fundamental Programming}
	Lorem Ipsum
	\subsection{Tujuan dari Fundamental Programming}
	\subsection{Topik Pengajaran Fundamental Programming}
	
\section{Proses Pengembangan Perangkat Lunak}
	\subsection{Definisi Pengembangan Perangkat Lunak}
	Lorem Ipsum
	\subsection{Model Pengembangan Perangkat Lunak}
	\subsection{Waterfall Model}
	\subsection{Data Flow Diagram}
