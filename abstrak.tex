%
% Halaman Abstrak
%
% @author  Andreas Febrian
% @version 1.00
%

\chapter*{Abstrak}

\vspace*{0.2cm}

\noindent \begin{tabular}{l l p{10cm}}
	Nama&: & Tegar Aldina Galari \\
	Program Studi&: & \program \\
	Judul&: & \judul \\
\end{tabular} \\ 

\vspace*{0.5cm}

\noindent 
Berdasarkan survei yang dilakukan, banyak pelajar menyukai kegiatan bermain \textit{game}. Bahkan \textit{game} sering menjadi pengganggu dalam kegiatan belajar sehingga banyak pikiran tersebar bahwa \textit{game} menimbulkan dampak negatif. Oleh karena itu banyak penelitian mulai memfokuskan bagaimana \textit{game} yang bersifat \textit{addictive} dapat membantu kegiatan belajar mengajar. Penelitian ini memberikan sebuah metode pembelajaran berbasis video permainan yang akan membantu pelajar dalam memahami materi terkait. Penelitian ini diawali dengan mencari demografi dan \textit{requirement} dari pelajar yang akan dijadikan target penelitian. Setelah didapatkan \textit{requirement} maka dilanjutkan dengan pembuatan prototipe dengan mempertimbangkan tampilan antarmuka dan interaksi untuk pengguna. Dilakukan evaluasi dari prototipe yang dibuat dengan menggunakan \textit{Usability Testing} pada 15 responden. Hasil evaluasi berupa persentasi keberhasilan responden melakukan perintah yang ada pada \textit{Usability Testing} sebesar 78,57\% (cukup-baik) dengan batas atas 93\% dan batas bawah 53\% (Sauro, 2011).

\vspace*{0.2cm}

\noindent Kata Kunci: \\ 
\noindent Pembelajaran, Pemrograman, \textit{Game}, \textit{Usability Testing}, \textit{Game Design}, antarmuka pengguna, interaksi pengguna, \textit{Addictive}\\ 

\newpage