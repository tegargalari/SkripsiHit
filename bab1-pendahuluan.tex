%-----------------------------------------------------------------------------%
\chapter{\babSatu}
%-----------------------------------------------------------------------------%
Pada bab ini dijelaskan latar belakang mengapa \saya melakukan peneltiian ini. Permasalahan, tujuan penelitian, batasan penelitian, dan sistematika penulisan dalam merancang penelitian ini akan dijelaskan oleh \saya pada bab ini.

%-----------------------------------------------------------------------------%
\section{Latar Belakang}
%-----------------------------------------------------------------------------%

Pada era teknologi seperti saat ini, teknologi informasi merupakan sebuah hal yang tidak bisa lepas dari kegiatan keseharian pada masyarakat. Hal ini terlihat dengan banyaknya kegiatan masyarakat yang menggunakan teknologi informasi sebagai alat bantu dalam mengerjaan pekerjaan mereka. Sebagai contoh seorang pegawai kantor menggunakan aplikasi telepon genggam seperti \textit{video game} untuk menghabiskan waktu disaat menunggu atau menghilangkan rasa bosan saat sedang istirahat.
\linebreak
\linebreak
Teknologi informasi begitu mudah didapatkan oleh masyarakat dan mempermudah dalam melakukan aktivitas. Menurut \ITTA (ITTA) teknologi informasi bertujuan sebagai pendukung dalam mengolah informasi dengan menggunakan perangkat keras maupun perangkat lunak. Tujuan tersebut bisa dikatakan berhasil dengan begitu populernya teknologi informasi dalam masyarakat karena banyak membantu pekerjaan masyarakat.
\linebreak\linebreak
Minat masyarakat terutama remaja sangat tinggi untuk mempelajari bagaimana cara membuat sebuah program. Hal ini terlihat pada data tahun 2013 hingga 2017 jumlah pendaftar pada jurusan  \program sangat tinggi. Namun tidak semua orang dapat dengan mudah mempelajari pemograman. Jika seseorang ingin mempelajari pemograman maka dia perlu mengetahui hal hal dasar tentang pemograman. Materi tersebut juga biasa di sebut dengan \textit{Fundamental Programming}.
\linebreak\linebreak
Christopher (2000) mengungkapkan \ddp merupakan sebuah latihan mengolah masalah logika matematika. \DDP bukan mengajarkan bagaimana cara menggunakan sebuah bahasa pemograman. Logika matematika membuat perkembangan pemograman jauh lebih cepat karena tidak terbatas akan penggunaan salah satu bahasa pemograman saja.
\linebreak\linebreak
\DDP diajarkan kepada mahasiswa Universitas Indonesia fakultas Ilmu Komputer. Dalam mempelajari \ddp terdapat beberapa kendala yang sering dihadapi. Kendalanya adalah materi, waktu, dan minat. Dari ketiga kendala tersebut materi dan waktu merupakan hal yang paling dominan. Hal ini didapatkan oleh penulis saat melakukan beberapa wawancara kepada mahasiswa yang mengambil \ddp. Materi yang sedikit diberikan oleh pembimbing dan sumber yang baik dari internet menjadi pokok masalah dari halangan materi. Waktu yang digunakan oleh mahasiswa dalam waktu satu minggu rata - rata hanya 5 - 6 jam saja. Hal ini juga tergantung oleh minat mahasiswa tersebut.
\linebreak\linebreak
Ada beberapa cara memecahkan solusi tersebut seperti ekperimental, pertemuan tatap muka, \textit{e-learning} dan \textit{game base learning}. \textit{Game base learning} merupakan sebuah metode pembelajaran menggunakan bantuan aplikasi permainan video. Aplikasi permainan video merupakan sebuah aplikasi yang memiliki banyak pro dan kontra didalam masyarakat. \textit{Video game} memiliki sebuah keuntungan dimana penggunanya dapat meningkatkan kemampuan yang berguna dalam kehidupannya. Seperti yang diutarakan oleh Lee (2014) terdapat beberapa kemampuan yang bisa didapat dari permainan video antara lain \textit{patience and perseverance}, \textit{forward thinking and strategic planning}, \textit{leadership and socialization}, mental dan \textit{creative prowess}, dan \textit{sympathy and empathy}. Meskipun memiliki keuntungan tersebut masyarakat masih memiliki pandangan bahwa sebuah permainan video merupakan pelaku utama tindak kenakalan yang dilakukan oleh anak mereka.
\linebreak\linebreak
Penyebab pandangan yang buruk karena masyarakat melihat sebuah permainan video hanya dari sebuah sudut pandang saja. Sebagai contoh sebuah permainan video dengan tema perang menampilkan tindak kekerasan dan saling bunuh antara pasukan. Hal ini menyebabkan muncul sebuah pandangan bahwa permainan video mengajarkan seseorang untuk bertindak kasar dan jahat kepada lawannya untuk mendapatkan hasil yang dia mau. Dalam permainan tersebut terdapat juga bagaimana cara mengelola sebuah negara, strategi, dan juga mengajarkan sejarah yang ada pada sekitar kita. Masyarakat pada umumnya sering menyalahkan permainan video sebagai penyebab dari tindak kejahatan yang terjadi disekitarnya, terlebih jika tindakan buruk tersebut dilakukan oleh pelajar.
\linebreak\linebreak
Hal tersebut menjadi salah satu beban pikiran pemerintah Indonesia. \kemkominfo (Kemkominfo) Indonesia telah membuat sebuah solusi dimana setiap \game yang beredar harus sesuai dengan kategori usia dan mencantumkan kategori tersebut dalam penjualan \game mereka. Seperti yang dijelaskan pada Peraturan Mentri No.11 \game dapat diklarifikasi sesuai dengan umur yaitu 3+, 7+, 13+, SU dan tidak dapat dikategorikan. Hal ini merupakan bentuk upaya agar \game memberikan dampak yang baik sesuai dengan perkembangan usia masing - masing penggunanya.
\linebreak\linebreak
Setelah adanya regulasi dari pemerintah terkait isu yang berkembang dimasyarakat diperlukan juga dukungan masyarakat selaku orang tua untuk membantu agar program yang dibuat oleh pemerintah ini sesuai dengan tujuannya. Para orang tua perlu melakukan bimbingan dan pengawasan pada anak mereka mengenai \game apa yang boleh dan tidak untuk mereka mainkan. Selain dapat mencegah dampak buruk yang terjadi , pengawasan kepada anak mereka akan membantu tumbuh kembang anak dan juga kemampuan yang sesuai dengan apa yang telah dijelaskan sebelumnya baik dalam fisik maupun pola pikir anak.
\linebreak\linebreak
pengembangan dan riset mengenai \game dalam bidang pendidikan di Indonesia sangat rendah jika dibandingkan dengan riset yang telah dilakukan pada negara maju. Pelaku industri dalam bidang pengembangan \game lebih memfokuskan diri mereka dalam memaksimalkan tingkat kesenangan pengguna dalam menggunakan atau memainkan \game mereka. Memaksimalkan kesenangan pengguna salah satunya dengan menaikan atau menurunkan tingkat kesulitan sesuai dengan kemampuan pengguna secara bertahan dan terstruktur, teknik ini biasa disebut dengan Flow. Hal tersebut dapat dilakukan dalam tahap \textit{game design}.
\linebreak\linebreak
Kirriemuir (2002) menemukan beberapa kendala dalam mengembangkan \game dalam bidang pendidikan. Hal yang mempersulit adalah melakukan identifikasi tentang apa saja komponen yang dibutuhkan dalam pengembangan sebuah \game untuk dunia pendidikan yang sesuai dengan kurikulum, melakukan sosialisasi kepada pihak terkait tentang keuntungan dalam menggunakan \game dalam proses belajar mengajar, kurangnya waktu untuk menerapkan metode pembelajaran dengan \game sehingga hasil yang diinginkan tidak dapat maksimal dari penggunakanya.
\linebreak\linebreak



%-----------------------------------------------------------------------------%
\section{Perumusan Masalah}
%-----------------------------------------------------------------------------%
Berdasarkan penjelsaan dalam latar belakang diperlukan analisis mengenai kondisi BRP dan cara belajar dengan melakukan pendekatan \textit{creative learning} melalui \game. Informasi dari analisis tersebut dan menjadi kerangka acuan utama untuk pengembangan prototipe \game yang selanjutnya akan dilakukan evaluasi untuk pengembangan prototipe selanjutnya.
\linebreak\linebreak
tujuan dari dilaksanakannya penelitian ini adalah mengetahui hal - hal apa saja yang dibutuhkan dan evaluasi mengenai \textit{game based learning} yang cukup memenuhi kopetensi sebagai bantuan pembelajaran dalam mata kuliah \ddp  pada topik \topik. Masalah yang akan dibahas meliputi :
\begin{itemize}
	\item Apakah pengembangan aplikasi \game berdasarkan prinsip - prinsip \textit{game design}
	\item Apa \textit{requirement} yang dibutuhkan untuk membuat aplikasi \game untuk mata kuliah \ddp (studi kasus potik \topik)
	\item Bagaimana hasil evaluasi aplikasi \game yang dikembangkan
\end{itemize}
Masalah tersebut akan menjadi pokok utama dan pencarian solusi dalam penelitian ini. Diharapkan hasil dari penelitian ini dapat menjawab pertanyaan tersebut dan menjadi salah satu rujukan dalam pengembangan konsep pembelajaran pada masalah tersebut. Tujuan dan manfaat dari penelitian ini akan dipaparkan pada subab selanjutnya.

%-----------------------------------------------------------------------------%
\section{Tujuan dan Manfaat Penelitian}
%-----------------------------------------------------------------------------%
Peneltian ini diharapkan mampu menghasilkan manfaat sebagai berikut
\begin{itemize}
	\item Berkontribusi dalam dunia pendidikan di Indonesia terutama dalam bidang pembelajaran Computer Science
	\item Mengurangi kesulitan mahasiswa dalam memahami materi sesuai dengan topik yang \saya bahas
	\item Pengenalan cara pembelajaran yang baru dalam memahami sebuah materi tertentu
	\item Mendapat kesempatan sebagai seorang \textit{game designer} dan langsung menciptakan sebuah game yang akan berguna bagi \saya dikemudian hari
\end{itemize}
Dalam mendapatkan tujuan tersebut, \saya mengalami keterbatasan dalam melakukan penelitian ini. Batasan - batasa yang \saya alami akan dipapatkan pada subab berikutnya.

%-----------------------------------------------------------------------------%
\section{Batasan Penelitian}
%-----------------------------------------------------------------------------%
Batasan yang dimiliki oleh penulis dalam mengerjakan penelitian ini sebagai berikut:
\begin{itemize}
	\item Proses penelitian dan pengembangan sistem menggunakan model \textit{waterfall} dan prototipe
	\item Hasil akhir pengembangan bukan merupakan sistem yang terprogram dengan rapih melainkan sebatas prototipe untuk menjukkan rancangan \textit{design} tantangan terpenuhi persyaratan
	\item Eksekusi proses pengembangan sistem dilakukan oleh \saya sendiri, tanpa tim dan pemangku kepentingan
	\item Dikarenakan keterbatasan waktu dan sumber daya \saya, partisipan wawancara memiliki ruang lingkup yang tidak jauh berbeda dengan \saya yaitu Fakultas Ilmu Komputer Universitas Indonesia. Evaluasi pun dilaksanakan spesifik pada mata kuliah \ddp
\end{itemize}
Dalam pengerjaannya \saya melakukan sesuai dengan sistematika yang ada untuk mendapatkan langkah - langkah yang sesuai dengan penulisan ilmiah. pada subab selanjutnya akan dilakukan penjelasan mengenai sistematika penulisan yang penulis lakukan
%-----------------------------------------------------------------------------%
\section{Sistematika Penulisan}
%-----------------------------------------------------------------------------%
Secara umum, laporan ini berisi mengenai perancanaan, pelaksaan, analisa data, rekomendasi yang diajukan, dan kesimpulan dari penelitian ini. Laporan ini terdiri tujuh bab utama dan disertai dengan sejimlah bagian pendukung. Laporan ini diawali dengan bab pendahuluan yang berisi latar belakang yang mendorong \saya melakukan penelitian ini, tujuan dan manfaat dari pelaksanaan ini, deskripsi batasan yang \saya alami dalam penelitian ini, dan sistematika penulisan laporan penelitian ini.
%-----------------------------------------------------------------------------%
Sistematika penulisan laporan adalah sebagai berikut:
\begin{itemize}
	\item Bab 1 \babSatu \\
	\item Bab 2 \babDua \\
	\item Bab 3 \babTiga \\
	\item Bab 4 \babEmpat \\
	\item Bab 5 \babLima \\
	\item Bab 6 \babEnam \\
	\item Bab 7 \babTujuh \\
\end{itemize}