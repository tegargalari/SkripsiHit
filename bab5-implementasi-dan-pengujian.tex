%-----------------------------------------------------------------------------%
\chapter{\babLima}
%-----------------------------------------------------------------------------%
Pada bab ini dibahas mengenai kesimpulan dari penelitian dan saran untuk penelitian
mendatang mengenai topik serupa.
%-----------------------------------------------------------------------------%
\section{Kesimpulan}
%-----------------------------------------------------------------------------%
Menjawab pertanyaan pertama pada rumusan masalah, dalam mengembangkan aplikasi video permainan harus mempertimbangkan beberapa hal. Pertama adalah jenis game yang akan dibuat, hal ini akan menjadi landasan dasar menentukan apa yang akan dilanjutkan selanjutnya. Kedua adalah menentukan elemen yang akan dibangun pada video permainan ini. Elemen tersebut adalah estetika, mekanik, naratif, dan teknologi. Dengan menentukan elemen ini maka semua syarat pembetukan sebuah game akan terpenuhi. Penentuan kedua hal tersebut akan didasari oleh pengguna dari aplikasi video permainan. Hal yang dibutuhkan untuk membentuk sebuah aplikasi video permainan untuk mata kuliah Dasar Dasar Pemrograman adalah sebuah rancangan sistem yang dapat memberikan nilai - nilai yang ada dalam tujuan mata kuliah Dasar Dasar Pemrograman. Memberikan sebuah materi tentang \textit{Computational Thinking} yang akan mengajarkan bagaimana sebuah logika komputer berjalan.
\linebreak\linebreak
Untuk menjawab pertanyaan kedua pada rumusan masalah, implementasi aplikasi video permainan untuk mata kuliah Dasar Dasar pemrograman mendapatkan respon yang cukup baik. Hasil kuantitatif untuk hasil evaluasi merupakan hasil dari UT yang dilakukan. Hasil dari UT adalah 78.57\% . Terdapat juga rekomendasi dari pengguna terkait aplikasi video permainan yang telah diterapkan. Hal tersebut akan terlihat jelas pada Subbab 4.6.

\section{Saran}
Saran yang akan diberikan merupakan pendapat dari penulis mengenai penelitian ini. Saran untuk penelitian ini adalah:
\begin{enumerate}
	\item Hasil penelitian ini diharapkan dapat berguna sebagai \textit{insight} untuk melakukan penelitan berikutnya, terutama penelitian yang terkait tentang pembelajaran menggunakan metode \textit{Game-Based Learning}
	\item Hasil \textit{Usability Testing} terutama bagian rekomendasi dapat menjadi dasar perbaikan apabila melakukan pengembangan aplikasi terkait maupun lanjutan.
	\item Mahasiswa dapat menemukan cara terbaik dalam mempelajari pemrograman karena telah mencoba metode ini.
	\item Peneliti lain dapat mengembangkan penelitian ini karena memiliki respon yang sangat baik dari mahasiswa Fakultas Ilmu Komputer.
	\item Materi pembelajaran \textit{Computational Thinking} dapat diajarkan sejak dini.
\end{enumerate}

