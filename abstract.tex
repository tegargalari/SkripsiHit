%
% Halaman Abstract
%
% @author  Andreas Febrian
% @version 1.00
%

	\chapter*{ABSTRACT}

\vspace*{0.2cm}

\noindent \begin{tabular}{l l p{11.0cm}}
	Name&: & \penulis \\
	Program&: & \programEng \\
	Title&: & \judulInggris \\
\end{tabular} \\ 

\vspace*{0.5cm}

\noindent 
Base on survey conducted, many students love playing game. Gaming distracts individuals from studying. it makes people think game have a negative impact. Therefore, researchers started the research about video game. Game is addictive. This study provides Game-Based Learning that will help the learner in understand related material. This study begins by looking for demographics and requirements of student who targeted for this research. Once found requirement, this study will making prototype by considering user interface, interaction and game design. Evaluated of the prototype made using Usability Testing. The result of evaluation is percentage of respondent's success while do the order in Usability Testing. Average complete rate is 78.57\% (good enough), upper limit 93\%, and lower limit 53\%.

\vspace*{0.2cm}

\noindent Keywords: \\ 
\noindent Learning, Programming, Game, Usability Testing, Game Design, User Interface, User Interaction, Addictive\\

\newpage