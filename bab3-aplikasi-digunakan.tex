%-----------------------------------------------------------------------------%
\chapter{\babTiga}
%-----------------------------------------------------------------------------%
Bagian ini menjelaskan metodologi penelitian termasuk di dalamnya pendekatan penelitian dan tahapan penelitian yang digunakan penulis dalam melakukan penelitian.
%-----------------------------------------------------------------------------%
\section{Pendekatan Penelitian}
%-----------------------------------------------------------------------------%

Penelitian ini penulis menggunakan pendekatan peneltian studi kasus. Cresswell (2013), pendekatan studi kasus adalah penelitian tentang suatu program, peristiwa, aktivitas, proses, atau kelompok individu.
\linebreak\linebreak
Pendekatan studi kasus yang dilakukan adalah proses pembelajaran mahasiswa fakultas ilmu komputer di Universitas Indonesia pada mata kuliah {\ddp}. Penelitian ini berusaha memahami perasaan responden terhadap suatu masalah. Penelitian ini dilakukan tanpa faktor - faktor eksternal sehingga tidak mempengaruhi pemikiran responden.

%-----------------------------------------------------------------------------%
\section{Tahapan Penelitian}
%-----------------------------------------------------------------------------%

Dalam penelitian ini penulis melakukan beberapa tahapan sesuai dengan Gambar 3.1.
\

%-----------------------------------------------------------------------------%
\subsection{Instrumen penelitian}
%-----------------------------------------------------------------------------%

Dalam pelaksanaan penelitian ini, dibutukan beberapa instrumen penelitian yang perlu dipersiapkan, antara lain
\begin{enumerate}
	\item Skenario \textit{usability testing} dan daftar wawancara untuk memudah pengguna
\end{enumerate}

%-----------------------------------------------------------------------------%
\subsection{Analisis dan Representasi Data}
%-----------------------------------------------------------------------------%

Dalam penelitian ini, penulis melakukan teknik data analisis \textit{simple qualitative analysis} dengan mengkategorisasikan data. Data yang berupa pilihan langsung oleh responden akan dihitung frequensi pemilihnya. Data yang berupa isian akan dikelompokan berdasarkan unik respon dari responden. 
\linebreak\linebreak
Hasil dari analisis evaluasi kemudian dilanjutkan ketahap implementasi sistem. Dalam tahapan ini, digunakan \textit{Eight Golden Rules of User Interface Design} yang dirumuskan oleh Shneiderman dan Plaisant(2010) dan digunakan dasar teori pengembangan \textit{Game Base Learning} oleh Kirriemuir (2002).
\linebreak\linebreak
Kedelapan aturan emas sebagai berikut.
\begin{itemize}
	\item \textit{Strive For Consistency}
	\subitem Konsistensi pada desain, tindakan, perintah dan bahasa.
	\item \textit{Cater To Universal Usability}
	\subitem Pemberian alternatif cara untuk pengguna dalam melakukan suatu hal. Hal seperti ini biasa disebut dengan \textit{shortcut}. Sehingga pengguna dapat lebih mudah dan cepat dalam menggunakannya.
	\item \textit{Offer Informative Feedback}
	\subitem Pemberian respon dari setiap aksi yang dilakukan oleh pengguna. Respon yang diberikan haruslah informatif dan dapat dimengerti oleh pengguna.
	\item \textit{Design Dialogs to Yield Closure}
	\subitem Membuat informasi dalam proses yang telah dilakukan oleh pengguna yang memuat banyaknya langkah yang harus ditempuh.
	\item \textit{Prevents Errors}
	\subitem Meminimalisasi terjadinya kesalahan saat pengguna menggunakan sistem.
	\item \textit{Permit Easy Reversal of Actions}
	\subitem Pemberian solusi yang mudah dimengerti dan cepat untuk pengguna apabila terjadi kesalahan.
	\item \textit{Support Internal Locus of Control}
	\subitem Menjadikan pengguna sebagai seseorang yang memegang penuh akan kontrol dalam sistem.
	\item \textit{Reduce Short-term Memory Load}
	\subitem Meminimalisasi hal yang harus diingat oleh pengguna saat menggunakan sistem.
\end{itemize}
Shneiderman dan Plaisant (2010) mengatakan depalan aturan emas ini telah dirumuskan sejak 1985, dan merupakan panduan dasar perancangan desain interaksi yang paling sering digunakan

