%-----------------------------------------------------------------------------%
\chapter{\babTiga}
%-----------------------------------------------------------------------------%
Bagian ini menjelaskan metodologi penelitian termasuk di dalamnya pendekatan penelitian dan tahapan penelitian yang digunakan penulis dalam melakukan penelitian.
%-----------------------------------------------------------------------------%
\section{Pendekatan Penelitian}
%-----------------------------------------------------------------------------%

Penelitian ini penulis menggunakan pendekatan peneltian studi kasus. Cresswell (2013), pendekatan studi kasus adalah penelitian tentang suatu program, peristiwa, aktivitas, proses, atau kelompok individu.
\linebreak\linebreak
Pendekatan studi kasus yang dilakukan adalah proses pembelajaran mahasiswa fakultas ilmu komputer di Universitas Indonesia pada mata kuliah {\ddp}. Penelitian ini berusaha memahami perasaan responden terhadap suatu masalah. Penelitian ini dilakukan tanpa faktor - faktor eksternal sehingga tidak mempengaruhi pemikiran responden.

%-----------------------------------------------------------------------------%
\section{Tahapan Penelitian}
%-----------------------------------------------------------------------------%



%-----------------------------------------------------------------------------%
\subsection{Studi literatur}
%-----------------------------------------------------------------------------%


%-----------------------------------------------------------------------------%
\subsection{title}

